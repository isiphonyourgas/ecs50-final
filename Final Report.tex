\documentclass[12pt,letterpaper]{article}
\usepackage{verbatim}
\usepackage{appendix}
\usepackage{fancyvrb}
\usepackage{relsize}
\title{The Maze Inside the Machine}
\author{Aaron Okano, Jason Wong, Meenal Tambe, \\ and Gowtham Vijayaragavan}
\begin{document}
\setcounter{page}{1}
\maketitle

The code that we decided to use was a C program titled maze.c. 
In the program, a two-dimensional array was placed into a file. 
An ``O'' indicated an open space while an ``X'' represented a closed 
portion in the maze. The goal of the program was to find the best 
possible path to complete the maze. The method for finding the 
best path was written into the find\_path function, which had 
recursive properties since it only called either itself or the 
{\bf printf()} function. When running the program, the coordinates 
for the best path were printed in row-major order. Because of 
its frequent dependence on two-dimensional arrays, maze.c was 
the best example to analyze the differences from its optimized 
and unoptimized source code files. After compiling maze.c, two 
.s files were made using the -S option for the unoptimized file 
and the -O3 option for the optimized files. The attached files 
were titled maze-opt.s for the optimized file and maze-noopt.s 
for the unoptimized file. One of the key differences that we 
noticed in the two files were the order of the functions and the 
number of jumps in the two files.

One of the most noteworthy changes in the optimized code is 
the new ordering. Many of the instructions that follow 
immediately after another instruction are placed right below it, 
so that the code may run smoothly. The optimized code also 
combines some functions that were separate in the unoptimized code. 
Instead of separating the functions, the optimized code reduces 
the number of jumps. In this regard, the amount of jumps is 
significantly reduced in the optimized code. The generous use of 
conditional jumps plays a big part in this. If a jump is not 
absolutely needed, the code just executes downward, which would not 
be possible if not for the appropriate ordering. A vast majority of 
the instructions in the unoptimized code end with the {\bf jmp}, 
whereas there are only 4 calls to {\bf jmp} in the optimized code. 
An instruction that clearly benefits from this is {\bf main()}:
\begin{verbatim}
main:
   pushl   %ebp
   movl    %esp, %ebp
   andl    $-16, %esp
   subl    $96, %esp
   cmpl    $2, 8(%ebp)
   movl    %ebx, 88(%esp)
   movl    %esi, 92(%esp)
   je      .L20
\end{verbatim}
A jump is made if there are only two arguments. Otherwise, the instruction 
after {\bf main()}, .L18, which performs necessary stack operations, 
and returns. It is extremely minimalist, and nothing else is performed 
unless the condition, that the number of arguments is two, is met.

Additionally, when comparing the two files, an observation we made 
was that the {\bf find\_path()} function in the optimized file was 
slightly longer in length than the {\bf find\_path()} function in the 
unoptimized file. It also appears that it is called quite often, with 
many checks and calculations being done inside {\bf find\_path()} itself. 
Then, after the jump from the function is made, the instructions do 
not get tangled in a series of jumps. If a jump is made, it is to 
terminate the program. Otherwise, the function, {\bf find\_path()}, is 
called again. This is in contrast to the unoptimized version, where 
jumps are made quite frequently from instruction to instruction.
        
The -O3 modification also makes some other very subtle changes. One of 
these subtle changes is how if statements are handled. When we look 
at the unoptimized version, we see that it it writes the function as if 
it was building the instructions based on the how the program would be 
during a straight run while in -O3 we see it more of a function based 
implementation. For example, take the first if statement checking 
whether or not there is a correct number of arguments. On the 
unoptimized version, we see that the jump is to occur if it is 
not equal. However, on the optimized version, we see that it jumps when 
it is equal to zero. This implementation can either hurt the run-time 
or help it. In our case it hurts it during {\bf main()} because it is 
doing an unnecessary jump. However, if we were searching for a very 
specific condition such as in our {\bf find\_path()}, looking for 
something that has a probability of less than one third of happening, 
then the optimized implementation is superior because it prevents excess jumps.

Another subtlety that we found was that the optimized version would push 
things straight into the stack. We can take our example from the time 
before the {\bf fopen()} call. The unoptimized version has
\begin{verbatim}
movl $.LC0, %edx
movl 12(%ebp), %eax
addl $4, %eax
movl (%eax), %eax
movl %edx, 4(%esp)
movl %eax, (%esp)
call fopen
\end{verbatim}
while the optimized version has the following:
\begin{verbatim}
movl $.LC1, 4(%esp)
movl 4(%eax), %eax
movl %eax, (%esp)
call fopen
\end{verbatim}
We see that if the last argument (.LC0 and LC1) is put directly into 
the stack in the optimized version while in the unoptimized version, 
we see that it is first put into EDX and then into the stack. Right 
there we can see that the optimized version is better. Why is it better? 
Because it is shorter. That means there are less instructions for 
the CPU which mean the run-time is faster and still reach the 
same desired outcome.

In addition to function calls, loops are also handled very differently 
in the two versions. In the unoptimized version we see that after each 
iteration of the loop, it jumps back to the top of the loop. In the 
optimized file, the .L20 function combined code for the .L3 and .L4 
from the unoptimized code. Instead of separating the functions, the 
optimized code reduces the number of jumps. .L20 plays a crucial role 
in making the "for" loop concise. In the unoptimized code, .L4 would 
be called whenever the "for" loops needed to run. The need for multiple 
jumps necessitates a counter. In the .L20 function of the optimized code, 
however, the need for a counter is eliminated since the following code 
for the "for" loops was merely inserted eight times:
\begin{verbatim}
   movl    %ebx, 8(%esp)
   movl    $11, 4(%esp)
   movl    %eax, (%esp)
   call    fgets
   movl    $7, 8(%esp)
\end{verbatim} 
This helps the program run faster in two ways. The first way it helps 
with runtime is that we do not have to check whether a loop has 
fulfilled its requirement. We can avoid that comparison thus saving 
CPU time. Also, if we avoid that comparison, there would be no need 
to jump. In this case, code compactness is sacrificed for speed.
   
As we traverse the program, we notice another subtlety. Every time 
we want to make a register 0, the unoptimized code moves 0 into the 
corresponding register (ie. \verb movl \  \verb $0, \  \verb %eax ).
The optimized version however, uses a different syntax. The function, 
.L8, uses the {\bf xorl} command in the optimized code:
\begin{verbatim}
  xorl    %eax, %eax
\end{verbatim}
This command provides a more concise way of using the {\bf movl} 
command since it doesn’t need to initialize the EAX register to 0 
first. This saves one line of instruction that would have used up space 
on the stack. The purpose of {\bf xorl} in this program is to initialize 
the for loop that was written in the maze.c code. {\bf xorl} shared 
similar properties with .L7 in the unoptimized code, where the EAX register 
was set to zero so that the recursive section could loop again without 
carrying values from the previous run. Moving a 0 into a register is 
considered an integer operation while {\bf xorl} is considered a bitwise 
operation and since bitwise operations are always faster than integer 
operations, using {\bf xorl} is more efficient. Overall, the reduction 
in the number of jumps and the increase in immediate code allows the 
program to run faster since this process does not need to move throughout 
the stack as frequently as the unoptimized file.

The reduction of jumps is not the only thing that affects the performance 
of the program.  The optimized version also takes advantage of the speed 
of registers to improve the performance of the program.  In the function 
{\bf find\_path()}, the unoptimized version, the index and the index2 
variables are compared from the stack while the optimized version copies 
the values from the stack into the registers.  Although it is an extra 
instruction, it pays off as the program progresses.  If it fails the first 
if statement (the compound if statement), it progresses to the second one.  
Here, index and index2 are accessed again.  In the optimized version, 
it simply uses the copied value in the registers instead of looking at 
the stack again.  Since registers are within the CPU, the access speed 
to registers are many magnitudes faster the from memory.  With the 
reduction of accesses to memory, the running time of the program 
is reduced thus improving the performance of the program further.

Because the optimized version places arguments to {\bf find\_path()} 
in registers, it is able to access the 2d array much more efficiently 
than the unoptimized code. To access a particular portion of the maze array, 
the computer needs to calculate maze+(index*8+index2). In the unoptimized 
version, the code appears as such:
\begin{verbatim}
   movl    12(%ebp), %eax
   sall    $3, %eax
   movl    %eax, %edx
   addl    8(%ebp), %edx
   movl    16(%ebp), %eax
   leal    (%edx,%eax), %eax
   movzbl  (%eax), %eax
   cmpb    $88, %al
\end{verbatim}
The values for the address of the maze pointer and the two index values 
need to be copied from the stack into registers to perform the necessary 
arithmetic operations. On top of that, there are still unnecessary 
operations, such as the MOV from EAX to EDX. The optimized code can 
take shortcuts because the arguments are already in registers:
\begin{verbatim}
   leal    (%edi,%ebx,8), %edx
   cmpb    $88, (%edx,%esi)
\end{verbatim}
Because of the easier access to the arguments, not only are the 
calculations easier to perform, the code can also take advantage 
of more advanced addressing modes.

One feature of GCC's optimizations that appears in maze-opt.s is the 
emphasis that is placed on safety. Encompassed within GCC's -O3 flag 
is the optimization \verb -fcaller-saves , which tells GCC to place 
the current register values at the front of the stack frame of the 
calling function. For example, surrounding the recursive calls 
to {\bf find\_path()} appears the code:
\begin{verbatim}
 movl %edx, -24(%ebp)
 movl %esi, 8(%esp)
 movl %edi, (%esp)
 call find_path
 movl -28(%ebp), %edx
\end{verbatim}
Here, the compiler put c(EDX) 24 bytes into the stack frame and then 
put it back into EDX after the find\_path call. The reason for this 
is to prevent situations such as the subroutine changing the EDX 
register or malicious leprechaun invasions.

We also observed GCC’s security emphasis in its use of the function 
{\bf \_\_printf\_chk()} in maze-opt.s in place of the usual {\bf printf()} 
that the unoptimized version uses. The optimized version uses the 
stack far more than the unoptimized one, so the need to check for 
stack overflows is more necessary. In particular, the {\bf printf()} 
function tends to use a tremendous amount of space on the stack, and 
added on top of the increased stack usage from optimizations such as 
\verb -fcaller-saves , the possibility of a stack overflow is increased. 
{\bf \_\_printf\_chk()} partially solves this problem by  checking the size 
of the stack prior to doing any stack-heavy calculations. Naturally, 
this extra action of checking the stack translates to slower performance, 
which illustrates how GCC does not focus entirely on the speed of the program.

Overall, the optimized version of maze.c was efficient, faster, and provided 
more safety than the unoptimized code. Although the code was longer and 
initially gave the illusion of a less organized and therefore less efficient 
way of decreasing the run-time of the program, the optimized file superseded 
this notion by providing a different approach to run the program and save memory.
By changing the order of instructions around significantly, reducing the number of 
jumps, and dealing with registers directly, the optimized code has traded conciseness 
for speed. However, when it comes to discovering the leprechaun’s pot of gold at 
the end of the maze, it is a worthwhile sacrifice. 
\newpage
\appendix .
\appendixpage
\section{Contributions}
\section{Code used in this report}
{\bf maze.c:}
\newline
\begin{Verbatim}[fontsize=\relsize{-2},numbers=left]
/* 
 * File:   maze.c
 * Author: Gowtham
 *
 * Created on July 17, 2010, 8:01 PM
 */

#include <stdio.h>
#include <stdlib.h>
#include <string.h>

//The purpose of this program is to scan in a file that has
//a grid of sorts, X being locations to which you cannot
//move to. Locations with an O are ok. The sequence of 
//positions to take to get from (0,1) to (7,7) is
//printed out. 

void find_path2(char maze[8][8], int index, int index2);
int find_path(char maze[][8], int index, int index2);
/*
 * 
 */
int main(int argc, char* argv[]) {
  FILE *inp;
  char maze[8][8];
  int i;
  if (argc == 2)
  {
    inp = fopen(argv[1], "r");

    for (i = 0; i < 8; i++)
    {
      fgets(maze[i], 11, inp);
    } //for

    if (find_path(maze, 7 , 7) == 0)
      printf("No path was found.");
    else
      printf("(7, 7)");

  } //if


  return 0;

} //main()

int find_path(char maze[][8], int index, int index2)
{
  if (index < 0 || index2 < 0 || index > 7 || index2 > 7)
  {
    return 0;
  } //if

  if (maze[index][index2]== 'X')
  {
    return 0;

  } //if

  if (index == 0 && index2 == 1)
  {
    return 1;
  } //if

  maze[index][index2] = 'X';


  if (find_path(maze, index, index2 + 1) == 1)
  {
    printf("(%d, %d) \n", index, index2 + 1);
    return 1;
  } //if
  if (find_path(maze, index, index2 - 1) == 1)
  {
    printf("(%d, %d) \n", index, index2 - 1);
    return 1;
  } //if
  if (find_path(maze, index - 1, index2) == 1)
  {
    printf("(%d, %d) \n", index - 1, index2);
    return 1;
  } //if
  if (find_path(maze, index + 1, index2) == 1)
  {
    printf("(%d, %d) \n", index + 1, index2);
    return 1;
  } //if



  return 0;
} //find_path()
\end{Verbatim}
\newpage
{\bf maze-noopt.s:}
\newline
\begin{Verbatim}[fontsize=\relsize{-2},numbers=left]
	.file	"maze.c"
	.section	.rodata
.LC0:
	.string	"r"
.LC1:
	.string	"No path was found."
.LC2:
	.string	"(7, 7)"
	.text
.globl main
	.type	main, @function
main:
	pushl	%ebp
	movl	%esp, %ebp
	andl	$-16, %esp
	subl	$96, %esp
	cmpl	$2, 8(%ebp)
	jne	.L2
	movl	$.LC0, %edx
	movl	12(%ebp), %eax
	addl	$4, %eax
	movl	(%eax), %eax
	movl	%edx, 4(%esp)
	movl	%eax, (%esp)
	call	fopen
	movl	%eax, 88(%esp)
	movl	$0, 92(%esp)
	jmp	.L3
.L4:
	leal	24(%esp), %eax
	movl	92(%esp), %edx
	sall	$3, %edx
	leal	(%eax,%edx), %edx
	movl	88(%esp), %eax
	movl	%eax, 8(%esp)
	movl	$11, 4(%esp)
	movl	%edx, (%esp)
	call	fgets
	addl	$1, 92(%esp)
.L3:
	cmpl	$7, 92(%esp)
	jle	.L4
	movl	$7, 8(%esp)
	movl	$7, 4(%esp)
	leal	24(%esp), %eax
	movl	%eax, (%esp)
	call	find_path
	testl	%eax, %eax
	jne	.L5
	movl	$.LC1, %eax
	movl	%eax, (%esp)
	call	printf
	jmp	.L2
.L5:
	movl	$.LC2, %eax
	movl	%eax, (%esp)
	call	printf
.L2:
	movl	$0, %eax
	leave
	ret
	.size	main, .-main
	.section	.rodata
.LC3:
	.string	"(%d, %d) \n"
	.text
.globl find_path
	.type	find_path, @function
find_path:
	pushl	%ebp
	movl	%esp, %ebp
	subl	$24, %esp
	cmpl	$0, 12(%ebp)
	js	.L7
	cmpl	$0, 16(%ebp)
	js	.L7
	cmpl	$7, 12(%ebp)
	jg	.L7
	cmpl	$7, 16(%ebp)
	jle	.L8
.L7:
	movl	$0, %eax
	jmp	.L9
.L8:
	movl	12(%ebp), %eax
	sall	$3, %eax
	movl	%eax, %edx
	addl	8(%ebp), %edx
	movl	16(%ebp), %eax
	leal	(%edx,%eax), %eax
	movzbl	(%eax), %eax
	cmpb	$88, %al
	jne	.L10
	movl	$0, %eax
	jmp	.L9
.L10:
	cmpl	$0, 12(%ebp)
	jne	.L11
	cmpl	$1, 16(%ebp)
	jne	.L11
	movl	$1, %eax
	jmp	.L9
.L11:
	movl	12(%ebp), %eax
	sall	$3, %eax
	movl	%eax, %edx
	addl	8(%ebp), %edx
	movl	16(%ebp), %eax
	leal	(%edx,%eax), %eax
	movb	$88, (%eax)
	movl	16(%ebp), %eax
	addl	$1, %eax
	movl	%eax, 8(%esp)
	movl	12(%ebp), %eax
	movl	%eax, 4(%esp)
	movl	8(%ebp), %eax
	movl	%eax, (%esp)
	call	find_path
	cmpl	$1, %eax
	jne	.L12
	movl	16(%ebp), %eax
	leal	1(%eax), %edx
	movl	$.LC3, %eax
	movl	%edx, 8(%esp)
	movl	12(%ebp), %edx
	movl	%edx, 4(%esp)
	movl	%eax, (%esp)
	call	printf
	movl	$1, %eax
	jmp	.L9
.L12:
	movl	16(%ebp), %eax
	subl	$1, %eax
	movl	%eax, 8(%esp)
	movl	12(%ebp), %eax
	movl	%eax, 4(%esp)
	movl	8(%ebp), %eax
	movl	%eax, (%esp)
	call	find_path
	cmpl	$1, %eax
	jne	.L13
	movl	16(%ebp), %eax
	leal	-1(%eax), %edx
	movl	$.LC3, %eax
	movl	%edx, 8(%esp)
	movl	12(%ebp), %edx
	movl	%edx, 4(%esp)
	movl	%eax, (%esp)
	call	printf
	movl	$1, %eax
	jmp	.L9
.L13:
	movl	12(%ebp), %eax
	leal	-1(%eax), %edx
	movl	16(%ebp), %eax
	movl	%eax, 8(%esp)
	movl	%edx, 4(%esp)
	movl	8(%ebp), %eax
	movl	%eax, (%esp)
	call	find_path
	cmpl	$1, %eax
	jne	.L14
	movl	12(%ebp), %eax
	leal	-1(%eax), %ecx
	movl	$.LC3, %eax
	movl	16(%ebp), %edx
	movl	%edx, 8(%esp)
	movl	%ecx, 4(%esp)
	movl	%eax, (%esp)
	call	printf
	movl	$1, %eax
	jmp	.L9
.L14:
	movl	12(%ebp), %eax
	leal	1(%eax), %edx
	movl	16(%ebp), %eax
	movl	%eax, 8(%esp)
	movl	%edx, 4(%esp)
	movl	8(%ebp), %eax
	movl	%eax, (%esp)
	call	find_path
	cmpl	$1, %eax
	jne	.L15
	movl	12(%ebp), %eax
	leal	1(%eax), %ecx
	movl	$.LC3, %eax
	movl	16(%ebp), %edx
	movl	%edx, 8(%esp)
	movl	%ecx, 4(%esp)
	movl	%eax, (%esp)
	call	printf
	movl	$1, %eax
	jmp	.L9
.L15:
	movl	$0, %eax
.L9:
	leave
	ret
	.size	find_path, .-find_path
	.ident	"GCC: (Ubuntu/Linaro 4.5.2-8ubuntu4) 4.5.2"
	.section	.note.GNU-stack,"",@progbits
\end{Verbatim}
\newpage
{\bf maze-opt.s:}
\begin{Verbatim}[fontsize=\relsize{-2},numbers=left]
	.file	"maze.c"
	.section	.rodata.str1.1,"aMS",@progbits,1
.LC0:
	.string	"(%d, %d) \n"
	.text
	.p2align 4,,15
.globl find_path
	.type	find_path, @function
find_path:
	pushl	%ebp
	movl	%esp, %ebp
	subl	$56, %esp
	movl	%esi, -8(%ebp)
	movl	16(%ebp), %esi
	movl	%ebx, -12(%ebp)
	movl	12(%ebp), %ebx
	movl	%edi, -4(%ebp)
	movl	8(%ebp), %edi
	testl	%esi, %esi
	js	.L8
	movl	%ebx, %eax
	shrl	$31, %eax
	testb	%al, %al
	jne	.L8
	cmpl	$7, %esi
	jg	.L8
	cmpl	$7, %ebx
	jg	.L8
	leal	(%edi,%ebx,8), %edx
	xorl	%eax, %eax
	cmpb	$88, (%edx,%esi)
	je	.L2
	cmpl	$1, %esi
	jne	.L12
	testl	%ebx, %ebx
	movb	$1, %al
	je	.L2
.L12:
	movb	$88, (%edx,%esi)
	leal	1(%esi), %edx
	movl	%edx, 8(%esp)
	movl	%edx, -28(%ebp)
	movl	%ebx, 4(%esp)
	movl	%edi, (%esp)
	call	find_path
	movl	-28(%ebp), %edx
	cmpl	$1, %eax
	je	.L14
	leal	-1(%esi), %edx
	movl	%edx, 8(%esp)
	movl	%edx, -28(%ebp)
	movl	%ebx, 4(%esp)
	movl	%edi, (%esp)
	call	find_path
	movl	-28(%ebp), %edx
	cmpl	$1, %eax
	je	.L14
	leal	-1(%ebx), %edx
	movl	%edx, 4(%esp)
	movl	%edx, -28(%ebp)
	movl	%esi, 8(%esp)
	movl	%edi, (%esp)
	call	find_path
	movl	-28(%ebp), %edx
	cmpl	$1, %eax
	je	.L16
	addl	$1, %ebx
	movl	%esi, 8(%esp)
	movl	%ebx, 4(%esp)
	movl	%edi, (%esp)
	call	find_path
	movl	%eax, %edx
	xorl	%eax, %eax
	cmpl	$1, %edx
	jne	.L2
	movl	%esi, 12(%esp)
.L13:
	movl	%ebx, 8(%esp)
	movl	$.LC0, 4(%esp)
	movl	$1, (%esp)
	call	__printf_chk
	movl	$1, %eax
	jmp	.L2
	.p2align 4,,7
	.p2align 3
.L8:
	xorl	%eax, %eax
.L2:
	movl	-12(%ebp), %ebx
	movl	-8(%ebp), %esi
	movl	-4(%ebp), %edi
	movl	%ebp, %esp
	popl	%ebp
	ret
	.p2align 4,,7
	.p2align 3
.L14:
	movl	%edx, 12(%esp)
	jmp	.L13
	.p2align 4,,7
	.p2align 3
.L16:
	movl	%esi, 12(%esp)
	movl	%edx, 8(%esp)
	movl	$.LC0, 4(%esp)
	movl	$1, (%esp)
	call	__printf_chk
	movl	$1, %eax
	jmp	.L2
	.size	find_path, .-find_path
	.section	.rodata.str1.1
.LC1:
	.string	"r"
.LC2:
	.string	"No path was found."
.LC3:
	.string	"(7, 7)"
	.text
	.p2align 4,,15
.globl main
	.type	main, @function
main:
	pushl	%ebp
	movl	%esp, %ebp
	andl	$-16, %esp
	subl	$96, %esp
	cmpl	$2, 8(%ebp)
	movl	%ebx, 88(%esp)
	movl	%esi, 92(%esp)
	je	.L20
.L18:
	xorl	%eax, %eax
	movl	88(%esp), %ebx
	movl	92(%esp), %esi
	movl	%ebp, %esp
	popl	%ebp
	ret
	.p2align 4,,7
	.p2align 3
.L20:
	movl	12(%ebp), %eax
	leal	16(%esp), %esi
	movl	$.LC1, 4(%esp)
	movl	4(%eax), %eax
	movl	%eax, (%esp)
	call	fopen
	movl	$11, 4(%esp)
	movl	%esi, (%esp)
	movl	%eax, %ebx
	movl	%eax, 8(%esp)
	call	fgets
	leal	24(%esp), %eax
	movl	%ebx, 8(%esp)
	movl	$11, 4(%esp)
	movl	%eax, (%esp)
	call	fgets
	leal	32(%esp), %eax
	movl	%ebx, 8(%esp)
	movl	$11, 4(%esp)
	movl	%eax, (%esp)
	call	fgets
	leal	40(%esp), %eax
	movl	%ebx, 8(%esp)
	movl	$11, 4(%esp)
	movl	%eax, (%esp)
	call	fgets
	leal	48(%esp), %eax
	movl	%ebx, 8(%esp)
	movl	$11, 4(%esp)
	movl	%eax, (%esp)
	call	fgets
	leal	56(%esp), %eax
	movl	%ebx, 8(%esp)
	movl	$11, 4(%esp)
	movl	%eax, (%esp)
	call	fgets
	leal	64(%esp), %eax
	movl	%ebx, 8(%esp)
	movl	$11, 4(%esp)
	movl	%eax, (%esp)
	call	fgets
	leal	72(%esp), %eax
	movl	%ebx, 8(%esp)
	movl	$11, 4(%esp)
	movl	%eax, (%esp)
	call	fgets
	movl	$7, 8(%esp)
	movl	$7, 4(%esp)
	movl	%esi, (%esp)
	call	find_path
	testl	%eax, %eax
	je	.L21
	movl	$.LC3, 4(%esp)
	movl	$1, (%esp)
	call	__printf_chk
	xorl	%eax, %eax
	movl	88(%esp), %ebx
	movl	92(%esp), %esi
	movl	%ebp, %esp
	popl	%ebp
	ret
	.p2align 4,,7
	.p2align 3
.L21:
	movl	$.LC2, 4(%esp)
	movl	$1, (%esp)
	call	__printf_chk
	jmp	.L18
	.size	main, .-main
	.ident	"GCC: (Ubuntu/Linaro 4.5.2-8ubuntu4) 4.5.2"
	.section	.note.GNU-stack,"",@progbits
\end{Verbatim}
\end{document}

